\documentclass[12pt,letterpaper]{article}

\usepackage{amsmath, amsthm, amsfonts, amssymb}
\usepackage{microtype, parskip, graphicx}
\usepackage[comma,numbers,sort&compress]{natbib}
\usepackage{lineno}
\usepackage{longtable}
\usepackage{docmute}
\usepackage{caption, subcaption, multirow, morefloats, rotating}
\usepackage{wrapfig}
\usepackage{hyperref}

\frenchspacing

\begin{document}

\section{Model selection and adequacy}

Our best model, as selected via LOOIC and WAIC, is the parameter rich ``past and vary'' model which includes our historical covariantes and allows all regression parameters to vary through time (Table \ref{tab:selection}). Our best model is an improvement over the next best model by approximately 13 LOOIC and 24 WAIC points. However, the actual differences between these models are minor as revealed by the wide the standard errors. These results mean that while our ``past and vary'' model could be considered the best of the four, all four of the models are very close and only offer incremental improvement but not major differences.
\begin{table}[ht]
  \centering
  \caption{Model comparison using WAIC and LOOIC}
  \begin{tabular}{ r r r r r }
    \hline
    Model & looic & looic se & waic & waic se \\
    \hline
    Past and vary & 12811.58 & 179.27 & 12807.61 & 170.21 \\
    No past but vary & 12834.53 & 179.07 & 12831.51 & 179.02 \\
    Past but no vary & 12836.19 & 179.20 & 12833.13 & 179.15 \\
    No past or vary & 12849.43 & 179.46 & 12847.08 & 179.43 \\
    \hline
  \end{tabular}
  \label{tab:selection}
\end{table}

Comparison of the in-sample AUC estimates between the four models reveals similar results as the selection criteria (Fig. \ref{fig:roc_hist}). While the parameter rich ``past and vary'' model has the greatest mean AUC when compared to the other three models, the other three models are quite similar. We chose to base our conclusions on the parameter rich ``past and vary'' model but these comparisons of model selection and adequacy illustrate the very small gain in performance that is achieved by allowing parameter effects to vary through time and including the historical covariates, though it appears that each of these choices on their own yield approximately equivalent models.
% ROC model comparison
\begin{figure}[ht]
  \centering
  \includegraphics[width=\textwidth,height=0.5\textheight,keepaspectratio=true]{figure/roc_hist}
  \caption{Posterior predictive AUC estimates for each of the four models being compared. These estimates are calculated from each of the models posterior predictive distribution compared to the empirical values. Models with a higher AUC values indicate better performance over models with lower AUC values. AUC is bounded between 0.5 and 1.}
  \label{fig:roc_hist}
\end{figure}

Additional comparison of in-sample model performance at each of the time intervals reveals just how similar our four models are to each other (Fig. \ref{fig:roc_ts}). There are few if any obvious differences between the models in their performances.
% ROC model comparison time series
\begin{figure}[ht]
  \centering
  \includegraphics[width=\textwidth,height=0.5\textheight,keepaspectratio=true]{figure/roc_ts}
  \caption{Comparison between the posterior predictive AUC estimates for each of the time intervals for each of the four models. These estimates are reflections of each model's ability to fit the various time intervals. The red line corresponds to the median AUC value, while the envelopes correspond to multiple credible intervals. In all cases, higher values AUC indicate greater performance versus lower AUC values.}
  \label{fig:roc_ts}
\end{figure}

Because of the parameter rich ``past and vary'' model displays some modest improvements over the other models we choose this model for all further analysis.

\section{Cross-validation}

The approximate out-of-sample predictive performance for our selected model, as measured by AUC, is approximately identical to our in-sample performance (Fig. \ref{fig:fold_auc}). The quality of performance, however, is not great, with average out-of-sample AUC estimated to be just above 0.7 which is far from perfect. This result means that while we expect our model to yield consistent results when provided with new data, we do not expect that our predictions will be very accurate.
% ROC OOS estimate
\begin{figure}[ht]
  \centering
  \includegraphics[width=\textwidth,height=0.5\textheight,keepaspectratio=true]{figure/fold_auc}
  \caption{Results from our five-fold cross-validation of the time-series. Each labeled distribution of AUC values correspond to expected out-of-sample performance as estimated from that fold. Each fold represents a section of data being predicted from a model fit to all data before the start of that fold. Given that there are only five folds, performance is measured from model predictions to 4 of the folds. Folds are labeled from oldest to youngest, with the oldest fold only being predicted by a single previous fold and the youngest fold being predicted by the four previous folds.}
  \label{fig:fold_auc}
\end{figure}

%When our cross-validation estimates are presented for each time interval, 
%% ROC OOS estimate time series
%\begin{figure}[ht]
%  \centering
%  \includegraphics[width=\textwidth,height=0.5\textheight,keepaspectratio=true]{figure/fold_auc_time}
%  \caption{Approximate out-of-sample AUC values calculated for each of the My intervals using five-fold cross-validation of the time series. The AUC of the individual My intervals within each fold is plotted to highlight the heterogentity in performance within and between folds. The AUC estimates from each fold are labeled and are numbered from oldest to youngest.}
%  \label{fig:fold_auc_time}
%\end{figure}


\section{Parameter estimates}

The overall average probability of a species going extinct during any interval is very low (Fig. \ref{fig:effect_est}). This result means that even if covariate effects are very large on the log-odds scale, when compared on the probability scale will only produce a minor change in probability of extinction. This is one of the hardest parts of logistic regression for people to understand. Effect sizes are on the log-odds scale but prediction is on the probability scale. When the intercept term is very negative or positive (greater than 2), covariate effects are fighting over an increasingly small amount of probability. Logistic regressions are easiest to interpret when the intercept term is approximately 0; this is a fundamental difficulty of dealing with very uneven data sets.

As expected, a species with greater than average geographic range is expected to have a greater probability of surviving than a species with average or less than average geographic range (Fig. \ref{fig:effect_est})

We estimated an overall positive effect of the change in geographic range on extinction probability; this means that a species gaining in geographic range between that observation and their previous observed geographic range is associated with a greater extinction risk than a species which decreased in geographic range (Fig. \ref{fig:effect_est}). This sign of this result is consistent with virtually all previous analyses of the relation between geographic range and extinction.

We estimated an overall positive effect of both global temperature and the lag of global temperature with species extinction probability (Fig. \ref{fig:effect_est}. This type of effect means that when global temperature is above the Cenozoic average, we would expect higher than average extinction risk. Similarly, if global temperature was above the Cenozoic average in the time interval before the current one, we would also expect higher than average extinction risk. This means that during a run of above Cenozoic average global temperatures (2+ intervals) we would expect an even greater extinction risk than if just the current interval is of above average temperature. These results imply that, on average, the species during Paleogene had a greater extinction probability that species during the Neogene.

% effects, average
\begin{figure}[ht]
  \centering
  \includegraphics[width=\textwidth,height=0.5\textheight,keepaspectratio=true]{figure/effect_est}
  \caption{Posterior estimates for the top-level covariate effects. These effect estimates are effectively the weighted average of the estimates for each of the time intervals, for each of the phyla analyzed. Presented are posterior densities with the median estimate labeled along 80\% credible intervals at the bottom of the density.}
  \label{fig:effect_est}
\end{figure}


The overall effect averages described above do not reflect the between time-interval variance of the effects. While the effect of geographic range has a consistent sign for the entire Cenozoic, the effects of the other covariates do not have a consistent sign (Fig. \ref{fig:effect_time_group}). 

% effects, time series
\begin{figure}[ht]
  \centering
  \includegraphics[width=\textwidth,height=0.5\textheight,keepaspectratio=true]{figure/effect_time_group}
  \caption{Effects of the four covariates as estimated for all time intervals and for each phyla. For each time series, the red line corresponds to the median estimate for each interval, while the envelopes represent various credible intervals as labeled.}
  \label{fig:effect_time_group}
\end{figure}

% variance components
\begin{figure}[ht]
  \centering
  \includegraphics[width=\textwidth,height=0.5\textheight,keepaspectratio=true]{figure/variance_components}
  \caption{Contribution of multi-level components to unmodeled variance. Larger values indicate a greater contribution to overall variance. Variance components labeled as ``overall'' represent variance between grouping factors, while those labeled  ``within groups'' correspond to variance within grouping factors. If the overall component is greater than the within component, then the groupings do not structure the overall variance. If the within component is greater than the overall component, then the overall variance is structured by the groupings.}
  \label{fig:variance_components}
\end{figure}

% risk estimate compared to change in geo-range


\end{document}
