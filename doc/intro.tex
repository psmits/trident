\documentclass[12pt,letterpaper]{article}

\usepackage{amsmath, amsthm, amsfonts, amssymb}
\usepackage{microtype, parskip, graphicx}
\usepackage[comma,numbers,sort&compress]{natbib}
\usepackage{lineno}
\usepackage{longtable}
\usepackage{docmute}
\usepackage{caption, subcaption, multirow, morefloats, rotating}
\usepackage{wrapfig}
\usepackage{hyperref}

\frenchspacing

\begin{document}

\section{Introduction}

Being able to predict which species are more likely to go extinct than others is critical for making good conservation decisions to limit the impact of the current biodiversity crisis. We cannot know, however, we do not yet know which species are going to go extinct because this has not happened yet -- it is unobservable. We approach this this problem by analyzing the past in order to predict the future. The fossil record preserves past extinction events, allowing us to develop a predictive model of species extinction based on this record and the properties of the observed species, both extinct and extant \citep{Harnik2012,Finnegan2015}. By assessing the predictive performance of this model on unobserved data, we can quantify how precise our best estimates will be for future extinctions. 

By studying how species vary in their extinction risk over time and we can assess which species are at greater risk under unobserved conditions. We that that extinction risk varies over time in both intensity (average rate) and selectivity (difference in risk between taxa) \citep{Payne2007,Payne2016k,Ezard2011}. Species, after all, can go extinct at any ``moment'' and the relative risk of extinction exhibited by different taxonomic groups and how that varies over time is an important dynamic which shapes the rate and structure of extinction. What has not been evaluated is as extinction intensity and selectivity change over time, how accurate are our assessments based on past events likely to be when applied to the future? By specifically including and modeling the temporal variation in extinction risk, we are able to improve our overall predictions because we incorporate and explicitly model differences between observations from across a range of possible intensities and selectivies

By analyzing extinction and survival data from the fossil record, the hope is this can aide in predicting the extinction risk of extant species -- after all, the present must at some level be a function of the past. Past paleobiological studies of extinction have frequently focused on identifying and measuring the effect of various predictors on extinction risk \citep{Harnik2011,Smits2015,Peters2008,Payne2007,Harnik2012,Exard2011,Foote2006} or on how to identify or measure these effects \citep{Alroy2010,Alroy2014,Alroy2001,Alroy2000,Alroy2000a,Foote2001}. This focus means that while we have a good understanding of which factors are strong and general determinates of extinction risk, we have less knowledge of how accurate or strong our predictions about the differences in extinction risk are. For example, while a predictor may be a ``significant'' factor in comparing the odds of extinction risk between groups, the practical difference that predictor makes on extinction can be minimal \citep{ARM}. By including the kinds of biological and abiotic predictors that have been shown to affect differences in extinction risk, we can quantify their actual, as opposed to relative, effects on predictive performance.

A related question is if the changes to biotic or abiotic predictors, and not just their values, are similarly important factors for predicting extinction. For example, we know that a species' global geographic range changes over its duration \citep{Foote2007,Liow2010,Liow2007,Kiessling2013}. We also know that a species' geographic range size is a good predictor of differences extinction risk \citep{Payne2007,Jablonski2003,Jablonski2008,Jablonski2006}. This begs the question: how does species' extinction risk change over their duration? While the phenomenon of species' geographic range change over time has been studied \citep{Foote2007,Liow2010,Liow2007,Kiessling2013}, the potential predictive impact of this change has been under-evaluated (but see \citet{Kiessling2013}. For example, does a species' extinction risk increase if they decreased in global geographic range size over 1 million years? Here, we explicitly model and quantify the effects of changing geographic range as well as differences in global climate on how well we can predict species extinction. Similarly, we include species geologic age at time of observation as a potential predictor of extinction -- a factor that may or may not contribute to differences in species extinction risk over time \citep{Smits2015,Finnegan2008,Ezard2012,VanValen1973,Liow2011,Crampton2016a}. Importantly, the inclusion of these ``historical'' predictors allows us to more fully evaluate the question of how much information about a species' past is necessary or useful when predicting a species' risk of extinction.

  For this kind of exercise, we chose to analyze what is the longest continuous and best resolved fossil record -- that of skeletonized marine plantonic microorganisms from the Cenozoic such as Foraminifera, Radiolarians, Diatoms, and calcareous nannofossils (e.g. coccolithophores). This data is available through the Neptune database, an online repository of species occurrences obtained through the Deep Sea Drilling Program and the Ocean Drilling Project \citep{Lazarus1994,SpencerCervato1999}. This database provides abundant samples in space and time, a high degree of temporal resolution for the entirety of the Cenozoic, and has an internally consistent taxonomic identification strategy -- as close to ideal data for this analysis as possible. Being able to analyze over 60 million years of fossil occurrences allows to actually quantify how accurate our predictions are in general, but also how much variation there is in predictive accuracy over time and in many different environmental contexts.

%In particular, there have been multiple major climatic and oceanographic events, both short and protracted in duration, over the course of the Cenozoic. For example, the Paleocene-Eocene Thermal Maximum (PETM) was an geologically close-to instantaneous global warming event believed to have been caused by a massive injection of carbon into the atmosphere over a 20,000 year period which rapidly warmed and is associated with substantial changes to the global distribution of species CITATION. In contrast, the Eocene-Oligocene boundary is associated with a protracted global cooling event caused by the slow opening of Drake's passage and the development of the Antarctic ice sheet CITATION. These moments of large scale global change are inherently ``different,'' so predicting extinction before, during, and after these is important -- especially if those events are analogous to future climate change.

Specifically, we might expect that our model's predictive performance is best during prolonged periods of similar stress, such as the Eocene-Miocene transition -- more samples from similar environments inherently improves future predictions in unobserved, but similar conditions. Alternatively, we would expect our model based predictions of extinction surrounding the PETM may be less accurate because there are inherently fewer samples from the climatic event.

Rarely are we able to analyze long periods of geological time at fine resolutions -- below the 5-10 My scale. Due to substantial effort and the unique biology of the system, the microfossil provides us the unique opportunity to analyze ecological and evolutionary patterns at approximately million-year time scales. Typical ``exceptional'' fossil records tend to be of individual taxonomic groups and for rarely longer than 10 million years. The Neptune database records multiple phyla-scale taxonomic groups for over 60 million years, with incredible temporal resolution supported by the various age-models of the deep-sea cores the occurrences are recorded from -- there is no equivalent fossil record. By analyzing patterns of extinction and global occurrence at fine temporal scales, we can better elucidate how well we can predict species extinction at human-relevant scales.




\end{document}
