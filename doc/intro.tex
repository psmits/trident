\documentclass[12pt,letterpaper]{article}

\usepackage{amsmath, amsthm, amsfonts, amssymb}
\usepackage{microtype, parskip, graphicx}
\usepackage[comma,numbers,sort&compress]{natbib}
\usepackage{lineno}
\usepackage{longtable}
\usepackage{docmute}
\usepackage{caption, subcaption, multirow, morefloats, rotating}
\usepackage{wrapfig}
\usepackage{hyperref}

\frenchspacing

\begin{document}

\section{Introduction}

Species extinction risk is strongly linked to geographic range size.

A species' geographic range size changes over time.

How does extinction risk change over a species duration? What effect does geographic range have on this?

We want to know if including information about past geographic range size improves our inference of species extinction risk.


A lot of work on species ``rise and fall'' dynamics has focused on the change in species geographic range over timeand what the shape of the average trajectory is CITATION. Issues that aren't addressed, like non-random drop out. Species can go extinct at any point in their ``lifetime'', so we might not observe a ``pure'' change to geographic range. We have to take this into account. The models that have been used do not do that which can totally mess-up analysis of geographic ranges over time. They need a joint-model, but that hasn't penetrated the field yet.

Additionally, lack of mechanism to explain a pure ``rise-fall''. Importantly, ``rise and fall'' is expected from a Markov process that has only a single absorbing boundary; at some point it will go back to 0. 

To demonstrate that species geographic range size is not a complete Markov process, it would be necessary to show that extinction risk depends not just on the current geographic range size, but the history of its geographic range size: has range size declined or increased between earlier and now?


We instead are focusing on how geographic range effects extinction risk of a species. We are not modeling geographic range through time, but instead modeling how species extinction is a function of geographic range over time. 




One of the great promises of paleobiology is that by studying the past we can better predict the future.  This promise is particularly pertinent given suggestions that risk assessments for some modern species could be improved by examining past extinction patterns and by using paleontological records to establish geographic range and abundance trajectories on geological timescales.  Any effort to assess future risk based on past extinctions and range trajectories must address two key questions:  (1) At a given timescale, are geographic range trajectories deterministic (past trends are likely to continue into the future) or Markovian (the future depends only on the present state)? (2) Given knowledge of past extinction/survival patterns and the present geographic ranges of extant taxa, how accurate are extinction risk predictions?  

To address these questions we analyze the fossil record of Cenozoic planktonic microfossil taxa (foramanifera, radiolarians, diatoms, and calcareous nanoplankton). Using a model of species survival, we analyzed how survival probability changes over time as a function of species age, time of observation, current geographic range, most recent change in geographic range, global temperature average, and the lag of global temperature. Our best supported model includes the historical covariates, change in geographic range and lag of global temperature, which indicates that the past improves our estimates of the present and future. 

The effects of the historical covariates are extremely small and vary considerably over time. For example, the effect of change in geographic range size can either increase or decrease probability of extinction depending on when in the Cenozoic the observation takes place. The improvement in predictive power by including these historical covariates is modest at best and reflects the rarity of extinction events (i.e. class-imbalance) and the extremely stochastic nature of species survival. Correcting for class imbalance, we find that in-sample model performance measures are approximately equal to out-of-sample performance as estimated from cross-validation. These results reflect the difficulty of estimating species extinction, and that while including historical covariates does improve model performance, that gain is very small.  This result implies that at million-year timescales geographic range trajectories are nearly Markovian, perhaps because the processes driving geographic range changes vary on substantially shorter timescales. The effect of change in geographic range on survival most likely stands for many interacting and unobserved processes which in-turn produce that species' geographic range and its affect on survival. 

Finally, we find support for species' extinction risk increases with age, though the strength of this effect varies among taxonomic groups. This effect is most pronounced in forams and radiolarians, and less pronounced in diatoms and calcareous nanoplankton. The greatest source of variance in survival probability is the timing of observation. Importantly, this result means that the time of an observation is a greater source of variation in survival probability when compared to species age. 

Ultimately, we find that including information on a species' change in geographic range size on average improves our predictions of species survival at million-year timescales. However, the effect of change in geographic range is much smaller than the effect of current geographic range, and highly variable through time as the effect changes sign and there are times where there is little evidence for any effect of past geographic range. The results of this study reinforce the importance of the promise of paleontology and using the past to predict the future.



how predictable is extinction? how much does the past improve our predictions? how much does the past improve our predictions?

we know that at almost all times, a species with geater than average geographic range is expected to have a lower extinction risk than a closely related species with a lower than average geographic range. what we don't know is if expanding or decreasing in geographic range over million-year time scales has any additional predictive information about a species' risk of going extinct.


\end{document}
