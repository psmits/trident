\documentclass[12pt,letterpaper]{article}

\usepackage{amsmath, amsthm, amsfonts, amssymb}
\usepackage{microtype, parskip, graphicx}
\usepackage[comma,numbers,sort&compress]{natbib}
\usepackage{lineno}
\usepackage{longtable}
\usepackage{docmute}
\usepackage{caption, subcaption, multirow, morefloats, rotating}
\usepackage{wrapfig}
\usepackage{hyperref}

\frenchspacing

\begin{document}

\section{Introduction}

Paleobiology is concerned with understanding why certain species go extinct while others do not CITATIOn. Being able to predict which species are more likely to go extinct than others is critical for making good conservation decisions to limit the impact of the current biodiversity crisis. We cannot know, however, we do not yet know which species are going to go extinct because this has not happened yet -- it is unobservable. One way of approaching this problem is the analyze the past and use that to predict the future. The fossil record preserves past extinction events, allowing us to develop a predictive model of species extinction based on this record and the properties of the observed species, both extinct and extant. By assessing the predictive performance of this model on unobserved data, we can quantify how precise our best estimates will be for future extinctions. 

By studying how species vary in their extinction risk over time and we can assess which species are at greater risk under unobserved conditions. We that that extinction risk varies over time in both intensity (average rate) and selectivity (difference in risk between taxa). Species, after all, can go extinct at any ``moment'' and the relative risk of extinction exhibited by different taxonomic groups and how that varies over time is an important dynamic which shapes the rate and structure of extinction. By including the kinds of biological and abiological predictors that have been shown to affect differences in extinction risk, we can improve our predictions of species extinction risk for a given set of conditions. Additionaly, by specifically including and modeling the temporal variation in extinction risk, we are able to improve our overall predictions because we incorporate and explictly model differences between observations from across a range of possible intensities and selectivies

Extinction studies have normally been focused on understanding how biological and abiological factors impact differences in extinction risk over time and between taxa CITATION.

Past studies have focused on identifying and measuring the effect of various predictors on extinction risk e.g. does difference in diet or geographic range affect differences in extinction risk?
Here we focus on the quality of our predictions e.g. how accurately can we predict if one species is more likely to go extinct than another?
increasing focus on prediction -- how accurate are we?
understanding versus prediction
long history focused on what determinates are important, but not on how accurate our predictions are given these factors.
the evaluate how well we make predictions in the present
trajectories studied before as phenonmenon, but not as tool -- we are using this information as tool!



has been suggested CITE seth wolfgang michael
incorporating extinction survival data from the fossil record can aide in predicting extinction risk of extant species -- the present must at some level be a function of the past
what has not been evaluated is given selectivity changes and intensity changes, how accurate are assessements based on the past likely to be in the future.





for this kind of excercise the best resolved fossil record of skeletonihze marine plantonic microorganisms (names). high temporal resolution, lots of samples, internally consistent taxonomy. 
Rich fossil record -- we might be able to better understand predicting extinction 
trajectories studied before as phenonmenon, but not as tool -- we are using this information as tool!
To address these questions we analyze the fossil record of Cenozoic planktonic microfossil taxa (foramanifera, radiolarians, diatoms, and calcareous nanoplankton). Using a model of species survival, we analyzed how survival probability changes over time as a function of species age, time of observation, current geographic range, most recent change in geographic range, global temperature average, and the lag of global temperature. 
cite the papers that evalutae the importance of the predcitors.
60 million year record lets us ask the 3 main things -- accurate, how much variation in accuracy, 


there have been major climate and oceanographic events that have prfoundly influenced the macroevolution of these groups CITATIONS.
intervals of time that are particularly 
things have changed over time -- these moments of big change are ``different'' so predicting them well is porentially important

the amoung of predictive power that the past has at any given 
just the last million years may matter -- so change to range matters
predictions to be better when similar stress over long period of time -- better on the average
a more protracted event/transition (Eo-Mi) -- might expect there to be more predictive power then because past might matter more (key is protraction)



\end{document}
