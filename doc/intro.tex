\documentclass[12pt,letterpaper]{article}

\usepackage{amsmath, amsthm, amsfonts, amssymb}
\usepackage{microtype, parskip, graphicx}
\usepackage[comma,numbers,sort&compress]{natbib}
\usepackage{lineno}
\usepackage{longtable}
\usepackage{docmute}
\usepackage{caption, subcaption, multirow, morefloats, rotating}
\usepackage{wrapfig}
\usepackage{hyperref}

\frenchspacing

\begin{document}

\section{Introduction}

Species extinction risk is strongly linked to geographic range size.

A species' geographic range size changes over time.

How does extinction risk change over a species duration? What effect does geographic range have on this?

We want to know if including information about past geographic range size improves our inference of species extinction risk.


A lot of work on species ``rise and fall'' dynamics has focused on the change in species geographic range over timeand what the shape of the average trajectory is CITATION. Issues that aren't addressed, like non-random drop out. Species can go extinct at any point in their ``lifetime'', so we might not observe a ``pure'' change to geographic range. We have to take this into account. The models that have been used do not do that which can totally mess-up analysis of geographic ranges over time. They need a joint-model, but that hasn't penetrated the field yet.

Additionally, lack of mechanism to explain a pure ``rise-fall''. Importantly, ``rise and fall'' is expected from a Markov process that has only a single absorbing boundary; at some point it will go back to 0. 

To demonstrate that species geographic range size is not a complete Markov process, it would be necessary to show that extinction risk depends not just on the current geographic range size, but the history of its geographic range size: has range size declined or increased between earlier and now?


We instead are focusing on how geographic range effects extinction risk of a species. We are not modeling geographic range through time, but instead modeling how species extinction is a function of geographic range over time. 





\end{document}
