\documentclass{letter}
\usepackage{microtype}
\frenchspacing

\signature{Peter D. Smits}
\address{Department of Integrative Biology \\ University of California Berkeley \\ 3040 Valley Life Sciences Building Rm. 5151 \\ Berkeley, CA 94720 \\ psmits@berkeley.edu}

\begin{document}
\begin{letter}{Editor \\ \textit{Philosophical Transactions of the Royal Society B: Biological Sciences}}
  \opening{Dear Editor,}


Please find enclosed our manuscript entitled ``How predictable is extinction? Forecasting species survival at million-year timescales'' by myself (Peter D. Smits) and Seth Finnegan. In this study we asked a relatively simple yet fundamental question for the emerging field of conservation paleobiology: how well do models conditioned on past observations predict future extinction events? To answer this question we analyzed the well-sampled fossil record of Cenozoic planktonic microfossil taxa (foramanifera, radiolarians, diatoms, and calcareous nanoplankton.) We examined how extinction probability varies over time as a function of species age, time of observation, current geographic range, change in geographic range, climate state, and change in climate state.

We used cross-validation to estimate the expected forecast performance of our model for future observations. Our models were found to have a 70-80\% probability of correctly forecasting the rank order of extinction risk for a random out-of-sample species pair, implying that determinants of extinction risk have varied only modestly through time. Including specific information about individual species trajectories resulted in only extremely marginal improvement in fit, suggesting that, at least at the million-year timescale, extinction risk is best described by dominantly Markovian (e.g. memoryless) models.

The relative quality and consistency of our models’ out-of-sample forecasting performance is encouraging given that these estimates are based on very limited biological and environmental information about the studied taxa. The results of this simple exercise bolster the case that fossil data can meaningfully inform present and future conservation decisions. We therefore believe that our study will be of general interest to paleontologists, conservation biologists, ecologists, and evolutionary biologists.  We look forward to your decision. Please send all correspondence regarding this manuscript to me via my email address (psmits@berkeley.edu) or that of my co-author, Seth Finnegan (sethf@berkeley.edu).

  \closing{Sincerely,}

  \encl{Article, supplementary text}.

\end{letter}
\end{document}

