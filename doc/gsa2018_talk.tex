\documentclass{beamer} 
\usepackage{amsmath,amsthm}
\usepackage{graphicx,microtype,parskip}
\usepackage{caption,subcaption,multirow}
\usepackage{attrib}
\usepackage{array}

\frenchspacing

\usetheme{default}
\usecolortheme{whale}

\setbeamertemplate{navigation symbols}{}

\setbeamercolor{title}{fg=blue,bg=white}

\setbeamercolor{block title}{fg=white,bg=gray}
\setbeamercolor{block body}{fg=black,bg=lightgray}

\setbeamercolor{block title alerted}{fg=white,bg=darkgray}
\setbeamercolor{block body alerted}{fg=black,bg=lightgray}


\title{How predictable is extinction?} 
\subtitle{Forecasting species survival at million-year timescales}
\author{Peter D Smits, Seth Finnegan}
\institute{Department of Integrative Biology, University of California -- Berkeley}
\date{}


\begin{document}


\begin{frame}
  \maketitle
\end{frame}


\begin{frame}
  \frametitle{Foundational assertion of conservation paleobiology }

  \begin{center}
    \begin{LARGE}
      By studying the \alert{past}, \\we can better predict the \alert{future}.
    \end{LARGE}
  \end{center}

\end{frame}


\begin{frame}
  \frametitle{What are we predicting?}

  \begin{center}
    \begin{LARGE}
      Extinction is \alert{hard} to predict, but is \alert{important} to conservation decisions.
    \end{LARGE}
  \end{center}

\end{frame}


\begin{frame}
  \frametitle{Predicting extinction}

  \begin{itemize}[<+->]
    \item A taxon with a \alert{greater than average} global geographic range is likely to \alert{survive for longer} than a taxon with \alert{less than average} global geographic range.
    \item A taxon's global geographic range can change over time.
    \item What happens to extinction risk as a taxon changes geographic range? How is extinction risk impacted if that taxon's global geographic range has recently \alert{increased} or \alert{decreased}?
  \end{itemize}

\end{frame}


\begin{frame}
  \frametitle{Data being analyzed}

  \begin{columns}
    \begin{column}{0.5\textwidth}
      \includegraphics[width=\textwidth,height=0.8\textheight,keepaspectratio=true]{../results/figure/occ_time_label}
    \end{column}
    \begin{column}{0.5\textwidth}
      \includegraphics[width=\textwidth,height=0.8\textheight,keepaspectratio=true]{../results/figure/age_label}
    \end{column}
  \end{columns}

\end{frame}


\begin{frame}
  \frametitle{How we're analyzing the data}

  \begin{itemize}%[<+->]
    \item Encoding the past
      \begin{itemize}
        \item Change in geographic range between current observation and previous observation.
        \item Average global temperature at time of previous observation (Mg/Ca isotope).
        \item Age in millions of years at time of observation.
      \end{itemize}
    %\item \alert{Compare} models using WAIC/LOOIC.
    \item Explore model adequacy using posterior predictive distribution.
    \item Estimate out-of-sample predictive performance using \(k\)-fold cross-validation.
  \end{itemize}

\end{frame}


\begin{frame}
  \frametitle{A conceptual model for predicting extinction}
      
  \begin{center}
    \includegraphics[width=\textwidth,height=\textheight,keepaspectratio=true]{figure/conceptual_diagram}
  \end{center}

\end{frame}


\begin{frame}
  \frametitle{Measuring performance: confusion matrix}

  \begin{center}
    \includegraphics[width=\textwidth,height=0.75\textheight,keepaspectratio=true]{figure/confusion_matrix_wiki}
  \end{center}
  
  \attrib{\footnotesize{wikimedia}}

\end{frame}
%
%
%\begin{frame}
%  \frametitle{Measuring performance: Receiver Operating Characteristic}
%
%  \begin{center}
%    \includegraphics[width=\textwidth,height=0.8\textheight,keepaspectratio=true]{figure/wiki_ROC_space-2}
%  \end{center}
%  
%  \attrib{\footnotesize{wikimedia}}
%
%\end{frame}


\begin{frame}
  \frametitle{Measuring performance: Receiver Operating Characteristic}
  
  \begin{center}
    \includegraphics[width=\textwidth,height=0.75\textheight,keepaspectratio=true]{figure/wiki_709px-ROC_curves}
  \end{center}
  
  \attrib{\footnotesize{wikimedia}}
  
\end{frame}

\begin{frame}
  \frametitle{Measuring performance: AUC ROC}

  \begin{columns}
    \begin{column}{0.4\textwidth}
      \includegraphics[width=\textwidth,height=\textheight,keepaspectratio=true]{figure/AUC}
  
      \attrib{\footnotesize{https://goo.gl/91nEpM}}
  
    \end{column}
    \begin{column}{0.6\textwidth}
      \[
        \text{AUC} = 
        \begin{cases} 
          0.5 & \text{non discrimination}\\
          0.6-0.7 & \text{poor} \\
          0.7-0.8 & \text{acceptable/fair} \\
          0.8-0.9 & \text{excellent/good} \\
          > 0.9 & \text{outstanding} \\
        \end{cases}
      \]

      The area represents the probability of correct ranking of a random ``extinct''-``extant'' pair.
    \end{column}
  \end{columns}

\end{frame}

\begin{frame}
  \frametitle{Measuring performance: \textit{k}-fold cross-validation}

  \begin{center}
    \includegraphics[width=\textwidth,height=0.8\textheight,keepaspectratio=true]{figure/ts_cv}
  \end{center}


  \attrib{\footnotesize{Ken Williams, https://goo.gl/qLcfL8}}
\end{frame}


% demonstrate as ROC curve?
\begin{frame}
  \frametitle{In-sample predictive performance, full dataset}

  \includegraphics[width=\textwidth,height=0.8\textheight,keepaspectratio=true]{../results/figure/auc_hist}

\end{frame}


\begin{frame}
  \frametitle{In-sample predictive performance, by time}

  \includegraphics[width=\textwidth,height=0.8\textheight,keepaspectratio=true]{../results/figure/auc_ts}

\end{frame}


\begin{frame}
  \frametitle{Cross-validation results, full dataset}

  \includegraphics[width=\textwidth,height=0.8\textheight,keepaspectratio=true]{../results/figure/fold_auc}

\end{frame}


\begin{frame}
  \frametitle{Cross-validation results, by time}

  \includegraphics[width=\textwidth,height=0.8\textheight,keepaspectratio=true]{../results/figure/fold_auc_time}

\end{frame}


\begin{frame}
  \frametitle{Summary}

  \begin{itemize}
    \item \alert{The past matters\dots} 
      \begin{itemize}
        \item Our best supported model includes our historical covariates and allows all effects to vary over time.
      \end{itemize}
    \item \alert{But not that much\dots}
      \begin{itemize}
        \item None of our models are good at predicting extinction.
      \end{itemize}
    \item Perhaps mechanisms behind changes to geographic range operate at sub-million year scales. Perhaps their effects are weak/masked at million (or greater) year scales.
  \end{itemize}

\end{frame}


\begin{frame}
  \frametitle{Acknowledgements}
  \begin{columns}
    \begin{column}{0.5\textwidth}
      \begin{itemize}
        \item \textbf{Seth Finnegan}
        \item Adiel Klompmaker 
        \item Emily Orzechowski
        \item Larry Taylor
        \item Sara Kahanamoku
        \item Josh Zimmt
        \item Franziska Franeck \\(University of Oslo)
      \end{itemize}
    \end{column}
    \begin{column}{0.5\textwidth}
      \begin{center}
        \includegraphics[height=0.15\textheight,width=\textwidth,keepaspectratio=true]{figure/github-logo}

        psmits.github.io/ \hspace*{0.05\textwidth} \textbf{trident}
      \end{center}
      \vspace*{0.02\textheight}
      \begin{center}
        \includegraphics[height=0.1\textheight,width=0.5\textwidth,keepaspectratio=true]{figure/twitter} 

        @PeterDSmits
      \end{center}
      \vspace*{0.02\textheight}
      \begin{center}
        \includegraphics[height=0.25\textheight,width=\textwidth,keepaspectratio=true]{figure/INAlogo}
      \end{center}
    \end{column}
  \end{columns}
\end{frame}

\end{document}

\end{frame}


\end{document}
